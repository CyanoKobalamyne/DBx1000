\section{Introduction}

As the world becomes more technology-oriented, the amount of data that is being stored and accessed has multiplied. From millions of users on Facebook to police records, the amount and complexity of data in virtual storage has incremented drastically. This growth has led to the study of \textbf{big data}, a term used for data sets that are too large and complex for traditional data processing methods. These large data sets are being analyzed in database management systems (DBMS). DBMS can be categorized into Online Transaction Processing (OLTP) and Online Analytical Processing (OLAP). OLTP databases are the main target of this paper and are characterized by a large number of short online transactions, such as bank transactions, flight ticket reservations, or hotel room bookings. OLAP, on the other hand, involves longer and more complex transactions such as business intelligence and data mining. \par

OLTP databases contain large amounts of important information. In order to preserve the information stored in the database in case of crashes, it is necessary to find a way to ensure the \textbf{durability} of the system. For this to happen, all of the data must be put on nonvolatile storage, or logged onto the disk. The most efficient way to put these large amounts of data onto the disk is still debated because there are many different ways to accomplish the task. \par

One of the first and most well-known algorithms designed for the logging process was the Algorithm for Recovery and Isolation Exploiting Semantics (AIRES). In this algorithm, trasactions are logged sequentially, with one transaction waiting until all others before it are comitted before it commits. More details on the serial logging algorithm can be found in section 2. With the introduction of new technologies, however, multi-core and multi-server system have become more prevalent. These new systems render the AIRES algorithm obsolete as we are no longer limited to using only one thread in the logging process, which can be a bottleneck as the number of transactions handled by the logger increases dramatically. This means that serial logging will no longer be useful in the future and new algorithms will be neccessary.  \par

The goal of this paper is to solve this problem and generate a new logging algorithm that is faster and more efficient on a system with a multi-cores and multi-servers. The two algorithms that are presented in this paper are batch logging and parallel logging. In batch logging, multiple loggers with independent local LSNs are used in place of a global LSN at the cost of some more latency. In parallel logging, we track the dependency between different transactions, which allows us to decide whether to log  independently or serially; multiple threads are employed simultaneously with their respective loggers and log files. \newline

The serial logging, batch logging, and parallel logging algorithms were implemented and then tested to determine the efficiency of each. \par


  
