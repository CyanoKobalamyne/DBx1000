\section{Serial Logging}

The state-of-the-art algorithm for serial logging is ARIES. In ARIES, each log record consists of one modified data tuple and the name of the transaction that modifies the tuple. It is assigned a log sequence number in ascending order. The log records are first pushed to volatile storage and then flushed to nonvolatile storage after the transaction is committed. Two data structures are maintained: the Dirty Page Table and the Transaction Table. The DPT keeps track of all the changes made to the database that have not been flushed to the disk, and the transaction table records all the transactions that are currently running in the system. During the recovery process, the system first recovers and updates the dirty page table and the transaction table, then recovers the system to the state immediately before the crash, and finally undoes all the transactions that have not been committed.
%Explain ARIES%


%Must give background (concurrency, other details)%

In our implementation of ARIES, we simplified the algorithm such that each transaction is logged together and corresponds to one unique LSN. Each transaction goes to log only after it is committed. This makes the dirty page table unnecessary, and we do not need to undo during the recovery process because only the committed transactions are reflected in stable storage.


\begin{figure}[!h]
  \includegraphics[width=\textwidth]{optimization.png}
  \caption{Serial Logging Optimization}
  \label{optimization}
\end{figure}\\
